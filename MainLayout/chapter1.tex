\renewcommand{\chaptername}{Chapter}
\chapter{Introduction}\label{chap1}

\jj{Beyond words themselves,} our speech carries \hl{para-verbal} information that conveys attitudes, emotions and intentions, often through prosody -- the \jj{so-called} \emph{melody} of speech \jj{(REF)}. Elements such as pitch, rate, and \st{intonation} play a crucial role in expressing \hl{emotions} and organizing communication. However, following a \jj{brain} stroke, particularly one affecting the right hemisphere, the ability to perceive or produce \hl{linguistic} and emotional prosody can be significantly impaired, hindering effective communication \jj{(REF)}. Although up to 54\% of patients with right hemisphere damage exhibit deficits in prosodic comprehension \jj{(REF)}, these impairments are often subtle and less noticeable than aphasia or motor dysfunction, resulting in underdiagnosis and insufficient treatment. Existing diagnostic tools, such as the Montreal Battery for the Evaluation of Communication (MEC \jj{REF}), provide \st{a systematic approach} \jj{simple thresholds of performance} but lack the sensitivity, specificity, and depth required to uncover the cognitive mechanisms underlying prosody deficits \jj{(REF)}. 

\jj{The goal of this thesis is to improve the diagnosis and comprehension of deficits of prosody perception after a brain stroke, by capitalizing on a \hl{recently-developping} psychophysical technique, \emph{reverse correlation} \jj{(REF)}. }

When studying the neural mechanisms that relate physical stimuli to perception, the modern field of psychophysics has indeed largely moved from simply measuring sensory thresholds and psychometric functions, and now provides a toolbox of techniques to measure and fit multi-staged models able to simulate participant behaviour. Notably for the example of speech prosody, the psychophysical technique of reverse-correlation (or “classification images” - \jj{REF MURRAY}) allows estimating, at the individual level, what sensory representations subtend the normal or abnormal perception of e.g. interrogative prosody \jj{(REF PONSOT)}.

\jj{In this work, we study the application of reverse correlation to model how controls and right-hemisphere stroke survivors differ in their processing linguistic prosody. Doing so, we identify a number of limitations of state-of-art techniques to analyse reverse-correlation data, which make them unsuitable for our patient population who differ from controls by their fatigability, low consistency, and a tendency to perseverate in their responses. We then introduce a number of novel algorithmic contributions to improve on these limitations, namely new techniques to estimate \emph{internal noise} (a model parameter determining response consistency) with non double-pass data, and a novel \emph{kernel} estimation procedure (based on the GLM-HMM architecture) to account for temporary states of perseveration. Finally, we apply these new techniques to a clinical dataset to study how reverse-correlation parameters agree and complement existing gold-standards for the diagnosis of prosodic impairment. \hl{+ add a sentence about studying perseveration if we have time to do that}}

\jj{This manuscript consists of 10 chapters, organized in 4 parts. Part I (Chapter 2 \& 3) reviews the biological and computational foundations for this work. Chapter 2 .... Chapter 3.... Part II etc. }

\st{In this chapter, we aim to expand this discussion by providing a detailed examination of prosodic deficits associated with right hemisphere damage and addressing the critical gaps in mechanistic understanding to advance research and clinical practice.\aynaz{have to talk about also reverse correlation if theoretical basis would include chap 1 \&2 maybe something like}}

