\renewcommand{\chaptername}{Chapter} 
\chapter{Alternative Approaches for Noise Estimation}\label{chap6}


\section {Double-pass}

practical limitations affect the applicability of Monte Carlo simulations in clinical settings. The estimation process requires large-scale simulations across a wide range of internal noise and bias values, making the approach computationally intensive. Running thousands of simulated trials to obtain stable estimates is feasible for offline analysis but becomes impractical for real-time clinical assessments where fast, efficient estimation methods are needed. Additionally, the method assumes that patient behavior conforms to the SDT framework, but in reality, some stroke survivors may demonstrate idiosyncratic decision-making strategies that fall outside of the model’s assumptions. This means that the estimated internal noise may not always reflect true perceptual variability but instead capture artifacts introduced by strategy shifts, attentional fluctuations, or cognitive rigidity.

\section {Confidence interval of GLM} 
\section {Intercept} 
\section {Ladislas method} 

