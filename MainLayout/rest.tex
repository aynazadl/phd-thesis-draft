%Prompt restoration of blood flow can mitigate or reverse these impairments. However, if circulation is not re-established in time, the damaged tissue dies and is eventually replaced by cerebrospinal fluid, leaving a permanent cavity.

%Finally, the "asymmetric sampling in time" hypothesis offers a unifying perspective, suggesting that prosodic elements characterized by long duration or pitch variation are right-lateralized, while those involving short duration or temporal variation are left-lateralized. This aligns with the dual-stream model of speech perception, where the RH specializes in slower, melodic changes, and the LH in rapid, temporal cues.

%It remains challenging to identify patients at risk of some of these impairments. Recent research suggests that stroke-related cognitive outcomes may be understood as the interaction between brain reserve (e.g., brain volume), cognitive reserve (e.g., education, cognitive-stimulation activities), and lesion load. This framework is is further supported by evidence linking it to functional stroke outcomes. (Gallucci et al 2024). 

%The findings of Garjardo-Vidal and colleagues, combined with results from other studies examining language comprehension in patients with right hemisphere damage, indicate that many patients are living with language comprehension deficits that remain unrecognized and untreated. In the future, patients with right hemisphere stroke should be routinely assessed for language, sarcasm, and emotional prosody deficits. Furthermore, providing education about these comprehension deficits can help stroke survivors and caregivers use compensatory strategies. These strategies could include caregivers and family explicitly stating the emotions they are feeling, and using simpler sentences rather than complex ones to help improve communication. Garjardo-Vidal and colleagues' work provides a foundation for future studies examining the specific executive functions that underlie language comprehension, and potentially for developing treatments that target these deficits.

%These challenges highlight the intricate interplay of neural networks responsible for processing prosody, auditory focus, and higher-level cognitive functions

%prosody is essential for conveying emotions, social cues, and nuanced meanings.

%Deficits in spoken discourse are a hallmark of communication impairments in individuals with RH stroke. Individuals with RH stroke are a heterogeneous group and have variable performance on discourse tasks. RH discourse has been described as egocentric, tangential, verbose or laconic, lacking emotional words, less informative, less relevant, confabulatory, and having poor global coherence (the ability to maintain a topic for the duration of conversation) compared to healthy controls (Agis et al., 2016; Blake, 2006; Joanette et al., 1986; Marini, 2012; Marini et al., 2005; Rogalski et al., 2010; Uryase et al., 1991). Such heterogeneity, poses a challenge for clinicians on objectively identifying discourse impairments. The cognitive–linguistic underpinnings for these deficits are not yet well understood. (Gajardo-Vidal et al. 2018)

%Converging evidence from EEG studies demonstrates that the auditory processing of lexical tones, at a pre-attentive stage, is lateralized to the right hemisphere (Luo et al., 2006; Ren et al., 2009), suggesting that the auditory processing of lexical tones is shaped mainly by acoustic properties at a pre-attentive processing stage(Chandrasekaran, Krishnan, Gandour, 2009b).

%\subsubsection{Extralinguistic Prosody} Extralinguistic prosody reflects the speaker’s characteristics, such as gender, age, or dialect. For example, a child’s voice is typically higher-pitched and lighter compared to an adult's voice.

%\subsubsection{Linguistic Prosody} Linguistic prosody contributes to the realization and structuring of linguistic levels, including lexical, syntactic, semantic, and pragmatic aspects. For example, when asking a question, the intonation rises at the end of the sentence. 
%\subsubsection{Emotional Prosody} Paralinguistic prosody conveys the speaker’s emotional state or mood by expressing through pitch, loudness, rate, and rhythm of speech (affective prosody) . For instance, a cheerful and dynamic voice with rising intonation may indicate joy.
#################################
% humans are experts in recognition tasks such as recognizing the emotion shown in a face. A central question in psychology is what sensory information humans use to achieve these senses.

%The Montreal Battery of Evaluation of Amusia (MBEA) is a key tool for assessing music perception, focusing on melodic organization tasks such as scale, contour, and interval tests. By examining pitch perception—a shared feature of both amusia and aprosodia—the MBEA offers valuable insights for diagnosing and treating these interrelated deficits.


###############


%Finally, the upper limit for internal noise estimation introduces additional concerns. When empirical values of (Pa, Pint1) approach the theoretical limit of infinite internal noise, the inference process breaks down. This can happen for two primary reasons: first, due to the limited number of repeated trials, which introduces sampling errors that artificially lower Pa, and second, due to the patient’s cognitive process diverging from the model assumptions, leading to behavioral patterns that the model cannot interpret correctly. As demonstrated in previous simulations, the number of trials required to reliably estimate high internal noise values increases exponentially, making the approach impractical for clinical applications. In cases where patients exhibit extremely low Pa values, the model struggles to differentiate whether this reflects true perceptual instability or merely a lack of task engagement, further reducing the reliability of the inferred estimates.
