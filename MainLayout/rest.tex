%Prompt restoration of blood flow can mitigate or reverse these impairments. However, if circulation is not re-established in time, the damaged tissue dies and is eventually replaced by cerebrospinal fluid, leaving a permanent cavity.

%Finally, the "asymmetric sampling in time" hypothesis offers a unifying perspective, suggesting that prosodic elements characterized by long duration or pitch variation are right-lateralized, while those involving short duration or temporal variation are left-lateralized. This aligns with the dual-stream model of speech perception, where the RH specializes in slower, melodic changes, and the LH in rapid, temporal cues.

%It remains challenging to identify patients at risk of some of these impairments. Recent research suggests that stroke-related cognitive outcomes may be understood as the interaction between brain reserve (e.g., brain volume), cognitive reserve (e.g., education, cognitive-stimulation activities), and lesion load. This framework is is further supported by evidence linking it to functional stroke outcomes. (Gallucci et al 2024). 

%The findings of Garjardo-Vidal and colleagues, combined with results from other studies examining language comprehension in patients with right hemisphere damage, indicate that many patients are living with language comprehension deficits that remain unrecognized and untreated. In the future, patients with right hemisphere stroke should be routinely assessed for language, sarcasm, and emotional prosody deficits. Furthermore, providing education about these comprehension deficits can help stroke survivors and caregivers use compensatory strategies. These strategies could include caregivers and family explicitly stating the emotions they are feeling, and using simpler sentences rather than complex ones to help improve communication. Garjardo-Vidal and colleagues' work provides a foundation for future studies examining the specific executive functions that underlie language comprehension, and potentially for developing treatments that target these deficits.

%These challenges highlight the intricate interplay of neural networks responsible for processing prosody, auditory focus, and higher-level cognitive functions

%prosody is essential for conveying emotions, social cues, and nuanced meanings.

%Deficits in spoken discourse are a hallmark of communication impairments in individuals with RH stroke. Individuals with RH stroke are a heterogeneous group and have variable performance on discourse tasks. RH discourse has been described as egocentric, tangential, verbose or laconic, lacking emotional words, less informative, less relevant, confabulatory, and having poor global coherence (the ability to maintain a topic for the duration of conversation) compared to healthy controls (Agis et al., 2016; Blake, 2006; Joanette et al., 1986; Marini, 2012; Marini et al., 2005; Rogalski et al., 2010; Uryase et al., 1991). Such heterogeneity, poses a challenge for clinicians on objectively identifying discourse impairments. The cognitive–linguistic underpinnings for these deficits are not yet well understood. (Gajardo-Vidal et al. 2018)

%Converging evidence from EEG studies demonstrates that the auditory processing of lexical tones, at a pre-attentive stage, is lateralized to the right hemisphere (Luo et al., 2006; Ren et al., 2009), suggesting that the auditory processing of lexical tones is shaped mainly by acoustic properties at a pre-attentive processing stage(Chandrasekaran, Krishnan, Gandour, 2009b).

%\subsubsection{Extralinguistic Prosody} Extralinguistic prosody reflects the speaker’s characteristics, such as gender, age, or dialect. For example, a child’s voice is typically higher-pitched and lighter compared to an adult's voice.

%\subsubsection{Linguistic Prosody} Linguistic prosody contributes to the realization and structuring of linguistic levels, including lexical, syntactic, semantic, and pragmatic aspects. For example, when asking a question, the intonation rises at the end of the sentence. 
%\subsubsection{Emotional Prosody} Paralinguistic prosody conveys the speaker’s emotional state or mood by expressing through pitch, loudness, rate, and rhythm of speech (affective prosody) . For instance, a cheerful and dynamic voice with rising intonation may indicate joy.
#################################
% humans are experts in recognition tasks such as recognizing the emotion shown in a face. A central question in psychology is what sensory information humans use to achieve these senses.

%The Montreal Battery of Evaluation of Amusia (MBEA) is a key tool for assessing music perception, focusing on melodic organization tasks such as scale, contour, and interval tests. By examining pitch perception—a shared feature of both amusia and aprosodia—the MBEA offers valuable insights for diagnosing and treating these interrelated deficits.


###############


%Finally, the upper limit for internal noise estimation introduces additional concerns. When empirical values of (Pa, Pint1) approach the theoretical limit of infinite internal noise, the inference process breaks down. This can happen for two primary reasons: first, due to the limited number of repeated trials, which introduces sampling errors that artificially lower Pa, and second, due to the patient’s cognitive process diverging from the model assumptions, leading to behavioral patterns that the model cannot interpret correctly. As demonstrated in previous simulations, the number of trials required to reliably estimate high internal noise values increases exponentially, making the approach impractical for clinical applications. In cases where patients exhibit extremely low Pa values, the model struggles to differentiate whether this reflects true perceptual instability or merely a lack of task engagement, further reducing the reliability of the inferred estimates.

%With only 50-150 trials, the double-pass method may not provide sufficient resolution to accurately estimate internal noise above a certain threshold (e.g., IN = 5).Monte Carlo simulations suggest that for values of IN > 5, increasing the number of trials (e.g., to 1000 or more) is necessary to get stable estimates.
%In stroke patients, where trials are often limited due to fatigue or cognitive impairment, the estimation method might not capture their true perceptual variability.
%When the number of double-pass trials is small (e.g., 50-150 trials), the variance in internal noise estimation increases.
%Confidence intervals widen significantly, making it difficult to differentiate noise values above a certain threshold.
%Empirical simulations show that IN estimates tend to be overestimated for values <6 and underestimated for values >6.
%Another major methodological limitation arises from trial count restrictions. Estimating internal noise requires a substantial number of repeated trials for statistical reliability, yet in clinical populations such as stroke patients, lengthy experimental sessions are not feasible due to fatigue and attentional constraints. As a result, the limited number of trials introduces a high degree of variability in the estimated internal noise, making the measurement less reliable. Additionally, the method does not account for response perseveration, a phenomenon frequently observed in stroke patients with right-hemisphere damage. Perseveration can artificially inflate estimates of internal noise because the model assumes that repeated trials are independent, whereas in reality, some patients persistently select the same response across multiple trials regardless of the presented stimulus. This means that some patients may appear to have high internal noise simply due to habitual responding rather than perceptual instability.
%A critical methodological limitation is that for participants exhibiting high internal noise, the estimated percentage of agreement (Pa) may approach chance levels, making it indistinguishable from completely random responses. In such cases, Monte Carlo simulations may fail to differentiate whether a patient genuinely exhibits pathologically high internal noise or if they are simply responding randomly due to task disengagement. The issue is exacerbated by the fact that, unlike low-level perceptual tasks where there is a correct or incorrect response, the present study involves a higher-level prosody classification task in which responses are subjective and based on an individual’s mental representation rather than an objective standard. This makes it even more challenging to attribute response variability solely to internal noise rather than to shifts in perceptual strategies.

